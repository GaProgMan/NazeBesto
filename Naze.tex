% The document contains a translation of the following title:
% 「上田次郎のなぜベストを尽くさないのか?」
% ISBN:978-4-05-402528-8
% Date of publishing: July 2004
% Number of pages: ~192
% Translation by Jamie Taylor (jamie@taylorj.org.uk) 2012

% Note: to compile this project, you may need to do a full
% install to TexLive (or a similar distribution).

%\documentclass[11pt, a4paper, twoside]{report}
\documentclass[paper=a4,fontsize=11pt]{report}	 	% KOMA-article class
\usepackage[margin=1in]{geometry}					% Drop the size of the margins to 1 inch
\textheight=700px									% Saving trees ;-) 

% Including the package "epigraph" this will allow quotes to
% be added at the begining of each chapter or section.
\usepackage{epigraph}

% The following two usepackage statements MUST be included
% for the Japanese text found in places throughout this
% document to parse correctly
\usepackage{CJKutf8}
\usepackage[overlap,CJK]{ruby}

% The following package enables LaTeX to include hyperlinks
% in the compiled document
\usepackage{url}

% Including following package enables the contents of
% non-TeX text files to be added to the compiled document
\usepackage{listings}

% pdfstartview=Fitforces the outputted pdf to open in a "fit
% to windows" zoom level.
% hidelinks tells pdflatex to NOT draw borders around any and
% all links in the compiled document
\usepackage[pdfstartview=Fit, hidelinks]{hyperref}

\begin{document}

% Tell LaTeX that we'll be using Chinese, Japanese and Korean
% characters in the following document and that it should be
%prepared for them
%\begin{CJK}{UTF8}{min}								% Use the 明朝 (Mincho) font for Japanese text
\begin{CJK}{UTF8}{goth}								% Use the ゴシック Gothic font for Japanese text
%\begin{CJK}{UTF8}{maru}							% Use the 丸ゴシック Maru Gothic font for Japanese text

% The above font commands (along with the following command)
% are adapted from the following URL:
% http://latex-my.blogspot.co.uk/2010/06/cjk-support-in-latex.html

%% By decreasing the \rubysep value, the Ruby characters will
% be placed further away from the characters they are above,
% making them easier to read
\renewcommand\rubysep{-0.1ex}

\title{Why Don't You Do Your Best?\\なぜベストを\ruby{尽}{つ}くさないのか}
\author{Ueda Jiro (Original)\\Jamie Taylor (Translation)}
\maketitle

% Inform TeX to parse the contents of each of the following
% files (preface, notes, etc.), placing their contents
% (sequentially) here
% This is the translated version of the Introduction chapter for 「上田次郎のなぜベストを尽くさないのか?」
% ISBN:978-4-05-402528-8
% Date of publishing: July 2004
% Number of pages: ~192
% Translation by Jamie Taylor (seppydude@gmail.com) 2012

\setcounter{footnote}{0}
\setcounter{endnote}{0}

% Ensure that this chapter is not added to the table of contents
\chapter*{Preface}

\epigraph{``Begin at the beginning," the King said gravely, ``and go on till you come to the end: then stop."}{--- \textup{Lewis Carroll}, Alice in Wonderland}

\section*{Legal Disclaimer}
This work is a translation of a copyrighted book. It is provided ``as-is'' and no copyright infringement was intended by the author. If you enjoy reading this translation, then please buy the original book. Performing a web search for the ISBN for the original book (4-05-402528-5) will yield a plethora of stores that sell physical copies of this book.

\section*{Licence Details}
% Store the licence details seperately so that they can be
% editted independent of this document
NazeBesto - a translation of the Japanese text ”上田次郎のなぜベストを尽くさないのか” (ISBN: 978-4-05-402528-8) Copyright \today{} Jamie Taylor
\par This program is free software: you can redistribute it and/or modify it under the terms of the GNU General Public License as published by the Free Software Foundation, either version 3 of the License, or any later version.
\par This program is distributed in the hope that it will be useful, but WITHOUT ANY WARRANTY; without even the implied warranty of MERCHANTABILITY or FITNESS FOR A PARTICULAR PURPOSE.  See the GNU General Public License for more details.
\par You should have received a copy of the GNU General Public License along with this program.  If not, see \href{http://www.gnu.org/licenses/}{http://www.gnu.org/licenses/}

\section*{Japanese Text}
This document will contain some Japanese characters - either in the footnotes or translation notes that are provided. If you have been able to open the document, then your PDF reader should be able to display them without any problems. However, if you find that there are sections of garbled characters, then you might need to install Asian character support for your reader.

\par Most kanji\endnote{Kanji are Chinese characters used in written Japanese. The translation of Kanji is ``Chinese Characters''} characters in this translation have ruby text\endnote{Ruby Text is the text placed near to a character to aid in phonetic pronunciation. More information on Ruby Text can be found here: \url{http://en.wikipedia.org/wiki/Ruby_text}} placed above them to aid the reader in the correct phonetic reading of the characters for the context in the given sentence.

\section*{Conventions Used}
Each chapter in the original book begins with a quote from the writer. This convention will be retained, but the quotes will be translated.
\par Footnotes will be supplied, where useful, to provide more information on a given sentence of phrase. These are provided in the hopes of giving a more fuller feel to the translation. Japanese is a language that does not translate fully to other languages, as there are nuances of meaning that are untranslatable, these nuances are expanded upon in the footnotes.

\theendnotes
% This is the translated version of the Introduction chapter for 「上田次郎のなぜベストを尽くさないのか?」
% ISBN:978-4-05-402528-8
% Date of publishing: July 2004
% Number of pages: ~192
% Translation by Jamie Taylor (seppydude@gmail.com) 2012

% The following is a last of escape sequences that can be used to create latin characters with accents and such
%\`{o}	ò	grave accent
%\'{o}	ó	acute accent
%\^{o}	ô	circumflex
%\"{o}	ö	umlaut or dieresis
%\H{o}	ő	long Hungarian umlaut (double acute)
%\~{o}	õ	tilde
%\c{c}	ç	cedilla
%\k{a}	ą	ogonek
%\l		ł	l with stroke
%\={o}	ō	macron accent (a bar over the letter)
%\b{o}	o	bar under the letter
%\.{o}	ȯ	dot over the letter
%\d{u}	ụ	dot under the letter
%\r{a}	å	ring over the letter (for å there is also the special command \aa)
%\u{o}	ŏ	breve over the letter
%\v{s}	š	caron/hacek ("v") over the letter
%\t{oo}	o͡o	"tie" (inverted u) over the two letters

% Ensure that the footnote count is reset for each chapter
\setcounter{footnote}{0}
\setcounter{endnote}{0}

\chapter*{A Note From The Translator}

\epigraph{``I gotta take notes when things occur to me"}{--- \textup{Jeff Bridges}, Unknown}

\section*{Reasons for Translating This Book}

Throughout the TRICK series (including the TV Show, the Specials and the Films), Ueda Jiro - the male lead - constantly refers to books that he has written. These books are based on his experiences solving paranormal crimes. These crimes are solved by Yamada Naoko - the female lead - working alongside Ueda. The books (``Why don't you do your best?''\endnote{(日本科学技術大学教授上田次郎のなぜベストを尽くさないのか?} (known as ``Naze Besto''), `Come over, Spiritual Phenomena''\endnote{日本科学技術大学教授上田次郎のどんと来い、超常現象 (単行本)} (known as ``Donto Koi'') , etc.) are used by supporting characters to point out how brave and smart Ueda is. The joke being that Ueda isn't, really, either. He's a physics professor at \ruby{東}{とう}\ruby{大}{だい}\endnote{\ruby{東}{とう}\ruby{京}{きょう}\ruby{大}{だい}\ruby{学}{がく} is often abbreviated to T\^{o} Dai}, but he often confuses basic magic tricks for paranormal phenomena. He is also very easily scared out of his wits.

\par When I found out that the books referenced in the show had been published, as a series of self-help style books, I decided that I would need to read them, as they would provide more insight into the characters. A cursory web-search showed me that none of the books had been translated, so I decided to order a copy of one of the books. I chose ``Naze Besto'' because it seemed to be used for more jokes in the series than ``Donto Koi''.


\section*{Who Are All These People?}

Trick is a Japanese comedy starring \ruby{阿}{あ}\ruby{部}{べ}\ruby{寛}{ひろし}\endnote{Abe Hiroshi} and \ruby{仲}{なか}\ruby{間}{ま}\ruby{由}{ゆ}\ruby{紀}{き}\ruby{恵}{え}\endnote{Nakama Yukie} as \ruby{上}{うえ}\ruby{田}{だ}\ruby{次}{じ}\ruby{郎}{ろう}\endnote{Ueda Jiro} and \ruby{山}{やま}\ruby{田}{だ}\ruby{奈}{な}\ruby{緒}{お}\ruby{子}{こ}\endnote{Yamada Naoko}, respectively.

\par 上田 is a professor of Physics at Tokyo University and uses logic and physics as a basis for his understand of the world and the people within it. Whereas 奈緒子 is a down on her luck, and (usually) out of work magician, who uses intuition and her magical training to see through most illusions.

\par 上田 is contacted by people to help them to understand or debunk paranormal phenomena (for instance: curses, good luck charms, psychic abilities and miraculous feats of strength). He usually chickens out when he witnesses said phenomena and implores 奈緒子 to help him explain what's happening.

\par 奈緒子 is insulted by many of the characters due to her small breasts. She usually fights back with a jibe about 上田's gigantic penis, or the fact that the wig that \ruby{矢}{や}\ruby{部}{べ}\ruby{謙}{けん}\ruby{三}{ぞ}\endnote{Yabe Kenzo} wears is has fallen off or looks obvious or out of place.

\par 矢部謙三 is an inept Assistant Detective, sent from Tokyo to investigate the crimes that are related to the phenomena that 上田 and 奈緒子 are trying to debunk. 矢部 (also the Japanese word for ``stop'') is played by \ruby{生}{なま}\ruby{瀬}{せ}\ruby{勝}{かつ}\ruby{久}{ひせ}\endnote{Namase Katsuhise}.

\par 奈緒子's Mother, 山田さとり\endnote{Yamada Satori} is a calligrapher who teaches the children of her home town about the power of the written word. She has an often hinted at but never explained, psychic connection with 奈緒子, and will often find a way of helping her daughter when she is in trouble. She is played by \ruby{野}{の}\ruby{際}{ぎわ}\ruby{陽}{よ}\ruby{子}{こ}\endnote{Nogiwa Yoko}.

\par More information about Trick can be found at the D-Additcs wiki page: \url{http://wiki.d-addicts.com/Trick}

\theendnotes

% Add a table of contents to the compiled document
\tableofcontents
% This is the translated version of the Introduction chapter for 「上田次郎のなぜベストを尽くさないのか?」
% ISBN:978-4-05-402528-8
% Date of publishing: July 2004
% Number of pages: ~192
% Translation by Jamie Taylor (seppydude@gmail.com) 2012

\chapter{Introduction}
Welcome!\footnote{Ueda writes in a very informal manner. The original opening is はじめに, which is quite informal}
\\These 13 characters\footnote {In the original Japanese, なぜベストを尽くさないのか is 13 characters and this is what Ueda is refering to} will change your life.
\\I find the existence of belief in witchcraft and sourcery to be a false faith.


% TODO: Add the contents of each chapter here

\end{CJK}

\end{document}