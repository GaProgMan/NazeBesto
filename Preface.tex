\chapter*{Preface}
%the above line ensures that the contents of this file are made into a chapter (called "Preface"), but that chapter is not added to the table of contents (that's what the * is for)
\subsubsection{Legal Disclaimer}
This work is a translation of a copyrighted book. It is provided ``as-is'' and no copyright infringement was intended by the author. If you enjoy reading this translation, then please buy the original book \footnote{Performing a web search for the ISBN will yield a plethora of online stores that sell hard copies of this book}.
\subsubsection{Japanese Text}
Needless to say, due to the nature of the original document, there will be some Japanese text peppered throughout this document. If you've gotten this far, then your pdf reader has already told you about the Japanese characters that are found in this document, usually in the form of footnotes. If the document reader program that you are using is a good one, then it will have already installed the necessary font packs required to display these characters (if they were missing from your computer to begin with).
\par Whenever ruby text\footnote{Ruby Text is the text placed near to Asian character to aid the reader in understanding which reading to use. More information on Ruby Text can be found here: \url{http://en.wikipedia.org/wiki/Ruby_text}} is used throughout this document, the exact reading is placed above the characters. This can, in some instance, lead to words that are made up of Hiragana and Katakana. This was done deliberately.
\par Most Asian characters have more than one reading, and they are separated (in Japanese) into three categories:
\begin{itemize}
\item Konyomi
\item Onyomi
\item Nanori
\end{itemize}
Dependant on the reading that is used, different characters are used for the ruby text. This all changes the way that the characters are read. The original reading for each character has been  included as the Ruby text, so that readers can learn the correct reading for each character in the given context.