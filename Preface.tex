% This is the translated version of the Introduction chapter for 「上田次郎のなぜベストを尽くさないのか?」
% ISBN:978-4-05-402528-8
% Date of publishing: July 2004
% Number of pages: ~192
% Translation by Jamie Taylor (seppydude@gmail.com) 2012

% Ensure that this chapter is not added to the table of contents
\chapter*{Preface}

\epigraph{``Begin at the beginning," the King said gravely, ``and go on till you come to the end: then stop."}{--- \textup{Lewis Carroll}, Alice in Wonderland}

\subsubsection{Legal Disclaimer}
This work is a translation of a copyrighted book. It is provided ``as-is'' and no copyright infringement was intended by the author. If you enjoy reading this translation, then please buy the original book. Performing a web search for the ISBN (4-05-402528-5) will yield a plethora of stores that sell physical copies of this book.

\subsubsection{Licence Details}
% Store the licence details seperately so that they can be
% editted independent of this document
NazeBesto - a translation of the Japanese text ”上田次郎のなぜベストを尽くさないのか” (ISBN: 978-4-05-402528-8) Copyright 2015 Jamie Taylor
\par This program is free software: you can redistribute it and/or modify it under the terms of the GNU General Public License as published by the Free Software Foundation, either version 3 of the License, or any later version.
\par This program is distributed in the hope that it will be useful, but WITHOUT ANY WARRANTY; without even the implied warranty of MERCHANTABILITY or FITNESS FOR A PARTICULAR PURPOSE.  See the GNU General Public License for more details.
\par You should have received a copy of the GNU General Public License along with this program.  If not, see \href{http://www.gnu.org/licenses/}{http://www.gnu.org/licenses/}

\subsubsection{Japanese Text}
This document will contain some Japanese characters - either in the footnotes or translation notes that are provided. If you have been able to open the document, then your PDF reader should be able to display them without any problems. However, if you find that there are sections of garbled characters in place of Japanese characters, then you might need to install Asian character support for your reader.

\par Most kanji\footnote{Kanji are Chinese characters used in written Japanese. The translation of Kanji is ``Chinese Characters''} characters in this translation (mostly proper nouns) have ruby text\footnote{Ruby Text is the text placed near to a character to aid in phonetic pronunciation. More information on Ruby Text can be found here: \url{http://en.wikipedia.org/wiki/Ruby_text}} placed above them to aid the reader in the correct phonetic reading of the characters. Most kanji have more than one reading, as such they are separated (in Japanese) into three categories:
\begin{itemize}
\item Konyomi
\item Onyomi
\item Nanori
\end{itemize}

\par Dependant on the contextual meaning of the kanji, different readings of the character are used for the ruby text.