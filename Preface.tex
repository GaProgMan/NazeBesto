\chapter*{Preface}
%the above line ensures that the contents of this file are made into a chapter (called "Preface"), but that chapter is not added to the table of contents (that's what the * is for)
\subsubsection{Legal Disclaimer}
This work is a translation of a copyrighted book. It is provided ``as-is'' and no copyright infringement was intended by the author. If you enjoy reading this translation, then please buy the original book\footnote{Performing a web search for the ISBN (4-05-402528-5) will yield a plethora of online stores that sell hard copies of this book}.
\par This project is protected by the GNU General Public Licence version 3 - details of this licence can by found in the following section.
\subsubsection{Licence Details}
NazeBesto - a translation of the Japanese text ”上田次郎のなぜベストを尽くさないのか” (ISBN: 978-4-05-402528-8) Copyright \today{} Jamie Taylor
\par This program is free software: you can redistribute it and/or modify it under the terms of the GNU General Public License as published by the Free Software Foundation, either version 3 of the License, or any later version.
\par This program is distributed in the hope that it will be useful, but WITHOUT ANY WARRANTY; without even the implied warranty of MERCHANTABILITY or FITNESS FOR A PARTICULAR PURPOSE.  See the GNU General Public License for more details.
\par You should have received a copy of the GNU General Public License along with this program.  If not, see \href{http://www.gnu.org/licenses/}{http://www.gnu.org/licenses/}
\subsubsection{Japanese Text}
This document will contain some Japanese characters - either in the footnotes or translation notes that are provided. If you have been able to open the document, then you PDF reader should be able to display them without any problems.
\par Whenever ruby text\footnote{Ruby Text is the text placed near to a character to aid in reading the character. More information on Ruby Text can be found here: \url{http://en.wikipedia.org/wiki/Ruby_text}} is used throughout this document, they provide reading instructions for complex or uncommon characters.
\par Most Kanji\footnote{Kanji are Chinese characters used in written Japanese. The translation of Kanji is ``Chinese Characters''} have more than one reading, and they are separated (in Japanese) into three categories:
\begin{itemize}
\item Konyomi
\item Onyomi
\item Nanori
\end{itemize}
\par Dependant on the reading that is used, different characters are used for the ruby text. This all changes the way that the characters are pronounced. The original reading for each character has been included as the Ruby text, so that readers can learn the correct reading for each character in the given context.